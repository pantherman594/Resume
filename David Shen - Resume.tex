%! TEX program = xelatex
\documentclass[10pt]{article}

\usepackage{geometry}
\usepackage{enumitem}
\usepackage{fontspec}
\usepackage{xifthen}
\usepackage{multirow}
\usepackage{xcolor}
\usepackage[none]{hyphenat}
\usepackage{fontawesome}

\usepackage[hidelinks]{hyperref}
\urlstyle{same} % style hyperlinks like regular text

\pagestyle{empty}
\geometry{margin=0.5in}
\setlength\parindent{0pt}
\setlist{nosep}
\setmainfont{Roboto-Light}
\setlength{\arrayrulewidth}{1.5pt}
\setlength{\tabcolsep}{10pt}

\newfontfamily\roboto{Roboto-Regular}
\newfontfamily\robotothin{Roboto-Thin}
\newcommand{\lmr}[1]{{\fontfamily{lmr}\selectfont#1}}

\newcommand{\name}[2]{
  \begin{minipage}{0.4\textwidth}
    \begin{flushright}
      \setlength{\tabcolsep}{0pt}
      \begin{tabular}{r}
        \fontsize{36pt}{0pt}\selectfont
        \MakeUppercase{#1 \textbf{#2}} \\
        \normalsize \, \textit{Software Engineer}\hfill \url{https://dav.sh/} \,
      \end{tabular}
    \end{flushright}
  \end{minipage}
}

\newcommand{\textlbf}[1]{{\roboto #1}}
\newcommand{\textlight}[1]{{\robotothin #1}}
\renewcommand{\date}[2]{#1 #2}
\newcommand{\daterange}[2]{#1 -- \ifthenelse{\equal{#2}{}}{\textit{Present}}{#2}}
\newcommand{\resumesection}[1]{\vspace{-0.2cm}\section*{#1}\vspace{-0.2cm}\vspace{0.1cm}}
\newcommand{\heading}[2]{\textbf{#1} \\ \textit{#2}}

\begin{document}

{
\setlength{\tabcolsep}{20pt}
\begin{center}
  \begin{tabular}{c | c}
% \name is self-explanatory. \topinfo is for things like social media, email, phone number, address, etc.
  \name{David}{Shen} & \begin{minipage}{0.4\textwidth}
    \vspace{6pt}
    \href{mailto:davi@d-shen.xyz}{davi@d-shen.xyz} \\
    \url{https://github.com/pantherman594} \\
    (617) 863-7436
    \vspace{6pt}
  \end{minipage}
\end{tabular}
\end{center}
}

\vspace{0.5cm}

\begin{minipage}[t]{0.25\textwidth}
  \begin{flushleft}
    \resumesection{Education}
    \textbf{Boston College} \newline
    \textit{B.S. in Computer Science} \newline
    \textlbf{GPA:} 3.92 \textlbf{Major GPA:} 4.0 \newline
    Dean's List First Honors. \newline
    \daterange{\date{Aug}{2018}}{\date{May}{2022}} \textit{\color{gray}{(exp.)}}

    \resumesection{Skills}
    \textbf{Languages} \newline
    C and C++ \newline
    TypeScript \newline
    Python \newline
    Rust \newline
    Java \newline
    Bash \newline
    HTML + CSS

    \vspace{0.25cm}

    \textbf{Technologies} \newline
    Linux \newline
    Git \newline
    Node.js \newline
    React + Redux \newline
    Deck.gl \newline
    Vue \newline
    Express \newline
    NGINX \newline
    MongoDB \newline
    MariaDB \newline
    PostgreSQL

    \resumesection{Awards}
    \textbf{Codestellation} \newline
    Judge's Pick \newline
    \date{Nov 10}{2019}

    \vspace{0.15cm}

    Best Web/Mobile Hack \newline
    \date{Nov 4}{2018}

    \vspace{0.25cm}

    \textbf{Hack the Heights} \newline
    First Place \newline
    \date{Apr 14}{2019}

    \vspace{0.25cm}

    \textbf{Hack Haverhill} \newline
    First Place \newline
    \date{Nov 17}{2018}

    \vspace{0.25cm}

    \textbf{BC Hacks} \newline
    Grand Prize; Hacker's Choice \newline
    \date{Nov 11}{2018}
  \end{flushleft}
\end{minipage}
\hfill
\begin{minipage}[t]{0.7\textwidth}
  \begin{flushleft}

  \resumesection{Experience}
  \textbf{Microsoft} \, Base OS Kernel \hfill Redmond, WA \, \textbf{Remote} \newline
  \textit{Software Engineering Intern} \hfill \daterange{\date{Jun}{2020}}{}
  \begin{itemize}
    \item Took ownership of a project bringing a feature highly desired by both internal Windows teams and external clients to Windows Containers, decreasing certain loading and setup scenarios from tens of minutes to seconds.
    \item Wrote a specification detailing the motivations behind and customer impact of the project.
  \end{itemize}

  \vspace{0.25cm}

  \textbf{Rocket Software} \, Zowe App Platform \hfill \textbf{Waltham, MA} \newline
  \textit{Software Engineering Intern} \hfill \daterange{\date{Jun}{2019}}{\date{Dec}{2019}}
  \begin{itemize}
    \item Built a new web interface for mainframes to filter and search for objects in a Db2 database, visualize and explore their hierarchical relationships, and define “Collections” of many objects.
    \item Designed and implemented a modular plugin system to add Microsoft Language Servers to a browser-based code editor for the mainframe.
  \end{itemize}

  \vspace{0.25cm}

  \textbf{Fellowship of Christian Clubs} \hfill \textbf{Remote} \newline
  \textit{Software Engineer} \hfill \daterange{\date{Jun}{2019}}{}
  \begin{itemize}
    \item Developed, maintained, and hosted the backend and database.
    \item Designed and helped develop the web interface.
  \end{itemize}

  \vspace{0.25cm}

  \textbf{rSpace} \, \textit{\color{gray}{Student Run Startup}} \hfill Atlanta, GA \, \textbf{Remote} \newline
  \textit{Co-Founder and CTO} \hfill \daterange{\date{Oct}{2018}}{}
  \begin{itemize}
    \item Planned, designed, and developed the entire backend and web frontend.
    \item Wrote up a comprehensive blueprint documenting the workings of the app, streamlining the development process.
  \end{itemize}

  \vspace{0.25cm}

  \textbf{Boston College} \hfill \textbf{Chestnut Hill, MA} \newline
  \textit{Teaching Assistant for Computer Science II} \hfill \daterange{\date{Jan}{2019}}{\date{May}{2019}}
  \begin{itemize}
    \item Wrote a script to automatically download students’ problem sets, suggest grades, and upload grades and comments to the Canvas Learning Management System, saving the professor hours of time.
    \item Assisted students with problem sets and answered student questions.
  \end{itemize}

  \vspace{0.25cm}

  \textbf{BC GET Delivery} \, \textit{\color{gray}{Student Run Startup}} \hfill \textbf{Chestnut Hill, MA} \newline
  \textit{Full Stack Developer} \hfill \daterange{\date{Oct}{2018}}{\date{Feb}{2019}}
  \begin{itemize}
    \item Developed an automated app-to-SMS messaging platform, to ease communication between deliverer and customer and improve customer privacy.
    \item Discovered and fixed critical security flaws.
  \end{itemize}

  \resumesection{Projects}

  \textbf{WikiWhere} \hfill \href{https://wikiwhere.org/}{\faLink \, \textit{https://wikiwhere.org/}} \quad \href{https://github.com/wikiwhere/wikiwhere}{\faGithub \, \textit{wikiwhere/wikiwhere}} \newline
  WikiWhere is a visualization of Wikipedia articles, and can be used to either traverse through linked articles or find the shortest path between two articles. The frontend uses d3 to visualize the graph, using data returned by the JavaScript and C++ backend. The backend queries Wikipedia data dumps and uses a bi-directional breadth first search to find the shortest path. I worked primarily on the backend and search algorithm.

  \vspace{0.25cm}

  \textbf{hackm.app} \hfill \href{https://hackm.app/}{\faLink \, \textit{https://hackm.app/}} \quad \href{https://github.com/hackmapp/hackmapp}{\faGithub \, \textit{hackmapp/hackmapp}} \newline
  hackm.app is an interactive map of the previous, current, and upcoming MLH hackathons. It scrapes the MLH website every hour using Python and BeautifulSoup, which is processed and served with an Express backend to be displayed in the website frontend using d3. I worked on the backend and helped with polishing the frontend.

  \end{flushleft}
\end{minipage}

\end{document}
